% Options for packages loaded elsewhere
\PassOptionsToPackage{unicode}{hyperref}
\PassOptionsToPackage{hyphens}{url}
%
\documentclass[
]{article}
\usepackage{lmodern}
\usepackage{amssymb,amsmath}
\usepackage{ifxetex,ifluatex}
\ifnum 0\ifxetex 1\fi\ifluatex 1\fi=0 % if pdftex
  \usepackage[T1]{fontenc}
  \usepackage[utf8]{inputenc}
  \usepackage{textcomp} % provide euro and other symbols
\else % if luatex or xetex
  \usepackage{unicode-math}
  \defaultfontfeatures{Scale=MatchLowercase}
  \defaultfontfeatures[\rmfamily]{Ligatures=TeX,Scale=1}
\fi
% Use upquote if available, for straight quotes in verbatim environments
\IfFileExists{upquote.sty}{\usepackage{upquote}}{}
\IfFileExists{microtype.sty}{% use microtype if available
  \usepackage[]{microtype}
  \UseMicrotypeSet[protrusion]{basicmath} % disable protrusion for tt fonts
}{}
\makeatletter
\@ifundefined{KOMAClassName}{% if non-KOMA class
  \IfFileExists{parskip.sty}{%
    \usepackage{parskip}
  }{% else
    \setlength{\parindent}{0pt}
    \setlength{\parskip}{6pt plus 2pt minus 1pt}}
}{% if KOMA class
  \KOMAoptions{parskip=half}}
\makeatother
\usepackage{xcolor}
\IfFileExists{xurl.sty}{\usepackage{xurl}}{} % add URL line breaks if available
\IfFileExists{bookmark.sty}{\usepackage{bookmark}}{\usepackage{hyperref}}
\hypersetup{
  pdftitle={Demovate-survey2020},
  pdfauthor={Sveinung Arnesen},
  hidelinks,
  pdfcreator={LaTeX via pandoc}}
\urlstyle{same} % disable monospaced font for URLs
\usepackage[margin=1in]{geometry}
\usepackage{graphicx,grffile}
\makeatletter
\def\maxwidth{\ifdim\Gin@nat@width>\linewidth\linewidth\else\Gin@nat@width\fi}
\def\maxheight{\ifdim\Gin@nat@height>\textheight\textheight\else\Gin@nat@height\fi}
\makeatother
% Scale images if necessary, so that they will not overflow the page
% margins by default, and it is still possible to overwrite the defaults
% using explicit options in \includegraphics[width, height, ...]{}
\setkeys{Gin}{width=\maxwidth,height=\maxheight,keepaspectratio}
% Set default figure placement to htbp
\makeatletter
\def\fps@figure{htbp}
\makeatother
\setlength{\emergencystretch}{3em} % prevent overfull lines
\providecommand{\tightlist}{%
  \setlength{\itemsep}{0pt}\setlength{\parskip}{0pt}}
\setcounter{secnumdepth}{-\maxdimen} % remove section numbering
\usepackage{amsmath}
\usepackage{booktabs}
\usepackage{caption}
\usepackage{longtable}

\title{Demovate-survey2020}
\author{Sveinung Arnesen}
\date{6/17/2020}

\begin{document}
\maketitle

\hypertarget{turistskatt}{%
\subsubsection{Turistskatt}\label{turistskatt}}

\begin{quote}
Det har vært oppe til diskusjon om man skal innføre skatt på
cruiseturismen i Bergen. Skatten skal gå til vedlikehold av fasiliteter
som turistene bruker. Hvor sterkt støtter eller motsetter du deg en slik
turistskatt? Svarskala 0 = Motsetter meg fullstendig - 10 = Støtter
fullstendig
\end{quote}

\captionsetup[table]{labelformat=empty,skip=1pt}
\begin{longtable}{lrrrr}
\caption*{
\large Oppslutning om turistskatt i Bergen\\ 
\small Regresjonsanalyse\\ 
} \\ 
\toprule
Variabel & Koeffisient & Standardfeil & t-verdi & p-verdi \\ 
\midrule
\multicolumn{1}{l}{Utdanning (Grunnskole som referansekategori)} \\ 
\midrule
Videregående & $0.92$ & $0.54$ & $1.70$ & $0.09$ \\ 
Universitet/høyskole (inntil 2 år) & $1.01$ & $0.62$ & $1.63$ & $0.10$ \\ 
Universitet/høyskole (mer enn 2 år) & $1.28$ & $0.53$ & $2.43$ & $0.02$ \\ 
\midrule
\multicolumn{1}{l}{Alder (Under 30 år som referansekategori)} \\ 
\midrule
30-44 år & $0.26$ & $0.27$ & $0.98$ & $0.33$ \\ 
45-59 år & $0.37$ & $0.27$ & $1.38$ & $0.17$ \\ 
60 år og eldre & $0.86$ & $0.26$ & $3.26$ & $0.00$ \\ 
\midrule
\multicolumn{1}{l}{Kjønn (Kvinne som referansekategori)} \\ 
\midrule
Mann & $-0.40$ & $0.18$ & $-2.22$ & $0.03$ \\ 
\midrule
\multicolumn{1}{l}{Bydel (Sentrum som referansekategori)} \\ 
\midrule
Arna og Åsane & $-0.51$ & $0.26$ & $-1.93$ & $0.05$ \\ 
Bergen Vest & $-0.49$ & $0.24$ & $-2.03$ & $0.04$ \\ 
Fana og Ytrebygda & $-0.24$ & $0.25$ & $-0.96$ & $0.34$ \\ 
\midrule
\multicolumn{1}{l}{\vspace*{-5mm}} \\ 
\midrule
Skjæringspunkt & $6.81$ & $0.62$ & $11.04$ & $0.00$ \\ 
Selvopplevd politisk innflytelse & $0.09$ & $0.04$ & $2.50$ & $0.01$ \\ 
Selvopplevd politisk innsikt & $-0.06$ & $0.03$ & $-1.70$ & $0.09$ \\ 
Selvplassering på politisk skala & $-0.18$ & $0.04$ & $-4.68$ & $0.00$ \\ 
\bottomrule
\end{longtable}

\includegraphics{Demovate-survey2020_files/figure-latex/cruiseturisme_fig-1.pdf}
\includegraphics{Demovate-survey2020_files/figure-latex/cruiseturisme_fig-2.pdf}
\includegraphics{Demovate-survey2020_files/figure-latex/cruiseturisme_fig-3.pdf}

\hypertarget{om-spuxf8rreundersuxf8kelsen}{%
\section{Om spørreundersøkelsen}\label{om-spuxf8rreundersuxf8kelsen}}

Respons Analyse har på oppdrag fra NORCE gjennomført en
innbyggerundersøkelse i Bergen i perioden 16. desember 2019 -- 13.
januar 2020. Undersøkelsen er gjennomført på telefon. Det er intervjuet
900 respondenter representativt for innbyggerne 18 år og eldre.

Spørreskjema er utarbeidet av NORCE og omhandler lokaldemokrati i Bergen
og inneholder bl.a. conjoint-spørsmål om deltakelse i borgerpanel i
Bergen. Intervjuet var beregnet til 10 minutters gjennomsnittlig
intervjutid. Det viste seg imidlertid at det tok litt lenger tid.
Gjennomsnittlig intervjutid viste seg å være 11,5 minutter.

Utvalget er trukket fra Bisnodes register over private telefonnumre i
Norge. Nedenfor har vi satt opp en oversikt over gjennomføringen og
frafallet i undersøkelsen. Vi har delt inn frafallet i følgende
kategorier:

\begin{itemize}
\item
  Ubesvart. Dette inneholder de vi har forsøkt å ringe uten at vi har
  fått kontakt med dem. Dette er i all hovedsak at telefonen ikke bli
  besvart når vi ringer, men i noen tilfeller dreier det seg om
  tilfeller der vi har fått gjort en avtale om tilbakeringing, men at
  vedkommende da ikke er tilgjengelig eller til stede likevel. Alle
  ubesvarte telefonnumre er lagt tilbake i utvalget og forsøkt oppringt
  igjen senere i intervjuperioden. Antall tilbakeringinger på de som
  havner i kategorien ubesvart varierer fra 2-12. Der fleste ligger
  imidlertid mellom 2-5 tilbakeringinger.
\item
  Nekt. Dette er personer vi har fått snakke med, men der denne ikke vil
  delta i undersøkelsen.
\item
  Utenfor målgruppen. Dette gjelder personer som det ikke lar seg gjøre
  å gjennomføre intervju på en faglig forsvarlig måte. Dette kan gjelde
  personer som ikke forstår norsk helt eller delvis, personer som hører
  dårlig, demente, psykisk utviklingshemmede osv.
\item
  Intervju. Dette er ingen frafallsgrunn, men vi har tatt det med for å
  beskrive hvordan bruttoutvalget fordelte seg under gjennomføringen.
  Dette er de som svarte på undersøkelsen.
\end{itemize}

På bakgrunn av dette kan vi oppsummere frafallet på denne måten.

\captionsetup[table]{labelformat=empty,skip=1pt}
\begin{longtable}{lr}
\toprule
Utfall & Antall \\ 
\midrule
Ubesvart & 6373 \\ 
Nekt & 4944 \\ 
Utenfor maalgruppen & 188 \\ 
Intervju & 900 \\ 
Sum brutto & 12405 \\ 
\bottomrule
\end{longtable}

Svarprosenten av de vi fikk kontakt med og som var i målgruppen er
således 15 \%. Dette er litt lavere enn vanlig, men kan skyldes at
omtrent halvparten av datainnsamlingen ble gjennomført i uken før jul.
Dette er tradisjonelt travle tider for folk, noe som nok kan gjøre at
svarvilligheten er lavere enn normalt.

\hypertarget{kodebok}{%
\subsection{Kodebok}\label{kodebok}}

Under følger en detaljert beskrivelse av hele datasettet slik det ble
levert fra Respons Analyse. Der kan man studere univariate fordelinger
på variablene i undersøkelsen; det vil si for eksempel hvor mange som
mener at skole og utdanning var svært viktig for deres stemme i
kommunevalget 2019,

\end{document}
